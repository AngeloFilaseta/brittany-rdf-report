\section{Design}

\centeredImage{img/brittany-rdf-graph.png}{Diagramma sommario delle classi e delle proprietà (Struttura Principale)}{0.8}

\noindent Il dominio è stato modellato utilizzando l'Ubiquitous Language appena illustrato. Sono stati inoltre ovviamente aggiunti elementi più di dettaglio e di basso livello, come le classi e le proprietà che fanno parte delle gerarchie "is a", più specifiche e per ora ancora mai nominate.

\noindent Questo grafico illustra solo le classi e le relazioni che fanno parte della struttura principale. In seguito vengono illustrate tutte le classi e tutte le proprietà. Le relazioni in giallo rappresentano le proprietà inverse di quelle in nero che hanno la freccia nel verso opposto.

\subsection{Concetti (Classi)}
Seguono gli identificatori delle classi create. Vengono riportate alcune note importanti. Dettagli più accurati ed approfonditi sono riportati nel file RDF \textit{brittany.ttl}:

\begin{itemize}
	\item \textbf{<\#NameRestriction>}:Classe di tipo \textit{owl:Restriction}, non anonima perché riutilizzata da più classi;

	\item \textbf{<\#Greenhouse>}:Utilizzato per modellare il concetto di Serra;

	\item \textbf{<\#Monitorable>}: Utilizzato per modellare il concetto di un "qualcosa" che può essere controllato da una classe di tipo \textit{<\#Edge>}. \textit{<\#Environment>} e \textit{<\#Plant>} sono sottoclassi di questa classe;

	\item \textbf{<\#Environment>}: Utilizzato per modellare il concetto di Ambiente;

	\item \textbf{<\#Plant>}: Utilizzato per modellare il concetto di Pianta;

	\item \textbf{<\#Edge>}: Utilizzato per modellare il concetto di Edge;

	\item \textbf{<\#Module>}: Utilizzato per modellare il concetto di Modulo;

	\item \textbf{<\#ComponentModule>}: Si tratta di una specializzazione di \textit{<\#Module>} che agisce necessariamente su dei componenti hardware (\textit{<\#ComponentHw>});

	\item \textbf{<\#DigitalLightModule>}: Una specializzazione di \textit{<\#ComponentModule>} che si occupa del sistema di illuminazione;

	\item \textbf{<\#OperationHandler>}: Utilizzato per modellare il concetto di Operation Handler;

	\item \textbf{<\#ComponentOperationHandler>}: Specializzazione di \textit{<\#OperationHandler>} che agisce necessariamente su dei componenti hardware (\textit{<\#ComponentHw>});

	\item \textbf{<\#ActionOperationHandler>}: Specializzazione di \textit{<\#OperationHandler>} utilizzata per definire un'operazione che effettua un'azione;

	\item \textbf{<\#PropertyOperationHandler>}: Specializzazione di \textit{<\#OperationHandler>} utilizzata per definire un'operazione che ricava una proprietà;

	\item \textbf{<\#DigitalLightHandler>}: Specializzazione di \textit{<\#ComponentOperationHandler>} per definire operazioni legate al sistema di illuminazione;

	\item \textbf{<\#IsOnDigitalLightHandler>}: Specializzazione di \textit{<\#DigitalLightHandler>} e di \textit{<\#PropertyOperationHandler>}. L'operazione si occupa di verificare se un determinato componente che fa parte del sistema di illuminazione è acceso.

	\item \textbf{<\#TurnOnDigitalLightHandler>}: Specializzazione di \textit{<\#DigitalLightHandler>} e di \textit{<\#PropertyOperationHandler>}. L'operazione si occupa di accendere un determinato componente che fa parte del sistema di illuminazione;

	\item \textbf{<\#TurnOffDigitalLightHandler>}: Specializzazione di \textit{<\#DigitalLightHandler>} e di \textit{<\#PropertyOperationHandler>}. L'operazione si occupa di spegnere un determinato componente che fa parte del sistema di illuminazione;

	\item \textbf{<\#ComponentHw>}: Utilizzato per modellare il concetto di Componente Hardware;

	\item \textbf{<\#SensorHw>}: Specializzazione di \textit{<\#ComponentHw>}. Un componente che ricava dati dall'area circostante;

	\item \textbf{<\#ActuatorHw>}: Specializzazione di \textit{<\#ComponentHw>}. Un componente che agisce sull'area circostante;

	\item \textbf{<\#DigitalLightHw>}: Specializzazione di \textit{<\#ActuatorHw>}. Un attuatore che si occupa del sistema di illuminazione;

	\item \textbf{<\#BrittanyThingDescriptor>}: Specializzazione di \textit{td:Thing}. Un Thing Descriptor che ha alcune proprietà sempre dello stesso tipo. Brittany utilizza infatti sempre alcuni tipi di valori e questa classe permette di porre quei vincoli.
\end{itemize}

\subsection{Relazioni (Proprietà)}
Seguono gli identificatori delle proprietà create. Vengono riportate alcune note importanti. Dettagli più accurati ed approfonditi sono riportati nel file RDF \textit{brittany.ttl}.

\definecolor{yellow(munsell)}{rgb}{0.65, 0.50, 0.0}
\newcommand\boldYellow[1]{\textcolor{yellow(munsell)}{\textbf{#1}}}

\subsubsection{ObjectProperty}
Vengono ora elencate le ObjectProperty, ossia le proprietà il cui range deve essere necessariamente un individuo.
\begin{itemize}
	\item \textbf{<\#hasEnvironment>}: Proprietà utilizzata per modellare il fatto all'interno di \textit{<\#Greenhouse>} c'è un \textit{<\#Environment>};

	\item \boldYellow{<\#isEnvironmentOf>}: Proprietà inversa di \textit{<\#hasEnvironments>};

	\item \textbf{<\#hasPlant>}: Proprietà utilizzata per modellare il fatto che in uno spazio è presente una  \textit{<\#Plant>};

	\item \boldYellow{<\#isPlantIn>}: Proprietà inversa di \textit{<\#hasPlant>};

	\item \textbf{<\#hasEnvironmentPlant>}: Specializzazione di \textit{<\#hasPlant>}, utilizzata per modellare il fatto che in un \textit{<\#Environment>} è presente una  \textit{<\#Plant>};

	\item \boldYellow{<\#isPlantInEnvironment>}: Proprietà inversa di \textit{<\#hasEnvironmentPlant>};

	\item \textbf{<\#hasGreenhousePlant>}: Specializzazione di \textit{<\#hasPlant>}, utilizzata per modellare il fatto che in una \textit{<\#Greenhouse>} è presente una  \textit{<\#Plant>};

	\item \boldYellow{<\#isPlantInGreenhouse>}: Proprietà inversa di \textit{<\#hasGreenhousePlant>};

	\item \textbf{<\#hasGreenhouseEdge>}: Proprietà utilizzata per modellare il fatto che in una \textit{<\#Greenhouse>} è presente un \textit{<\#Edge>};

	\item \boldYellow{<\#isEdgeInGreenhouse>}: Proprietà inversa di \textit{<\#hasGreenhouseEdge>};

	\item \textbf{<\#hasModule>}: Proprietà utilizzata per modellare il fatto che è presente un \textit{<\#Module>} all'interno di un \textit{<\#Edge>};

	\item \boldYellow{<\#isModuleIn>}: Proprietà inversa di \textit{<\#hasModule>};

	\item \textbf{<\#hasHandler>}: Proprietà utilizzata per modellare il fatto che esiste un \textit{<\#OperationHandler>} all'interno di un \textit{<\#Module>};

	\item \boldYellow{<\#isHandlerIn>}: Proprietà inversa di \textit{<\#hasHandler>};

	\item \textbf{<\#monitor>}: Proprietà utilizzata per modellare il fatto che un \textit{<\#Edge>} può essere posto in un qualsiasi \textit{<\#Monitorable>};

	\item \boldYellow{<\#isMonitored>}: Proprietà inversa di \textit{<\#monitor>};

	\item \textbf{<\#monitorPlant>}: Specializzazione di \textit{<\#monitor>} dedicata al monitor di un \textit{<\#Plant>};

	\item \boldYellow{<\#isPlantMonitored>}: Proprietà inversa di \textit{<\#monitorPlant>};

	\item \textbf{<\#monitorEnvironment>}: Specializzazione di \textit{<\#monitor>} dedicata al monitor di una \textit{<\#Environment>};

	\item \boldYellow{<\#isEnvironmentMonitored>}: Proprietà inversa di \textit{<\#monitorEnvironment>};

	\item \textbf{<\#hasComponentHw>}: Proprietà utilizzata per modellare il fatto che qualcosa possiede un \textit{<\#ComponentHw>};

	\item \boldYellow{<\#isComponentHwIn>}: Proprietà inversa di \textit{<\#hasComponentHw>};

	\item \textbf{<\#hasSensorHw>}: Specializzazione di \textit{<\#hasComponentHw>} utilizzata per modellare il fatto che qualcosa possiede un \textit{<\#SensorHw>};

	\item \boldYellow{<\#isSensorHwIn>}: Proprietà inversa di \textit{<\#hasSensorHw>};

	\item \textbf{<\#hasActuatorHw>}: Specializzazione di \textit{<\#hasComponentHw>} utilizzata per modellare il fatto che qualcosa possiede un \textit{<\#ActuatorHw>};

	\item \boldYellow{<\#isActuatorHwIn>}: Proprietà inversa di \textit{<\#hasActuatorHw>};

	\item \textbf{<\#hasDigitalLightHw>}: Specializzazione di \textit{<\#hasActuatorHw>} utilizzata per modellare il fatto che qualcosa possiede un \textit{<\#DigitalLightHw>};

	\item \boldYellow{<\#isDigitalLightHwIn>}: Proprietà inversa di \textit{<\#hasDigitalLightHw>};

\begin{info}[owns:]
	La parola chiave "owns" è stata utilizzata per contraddistinguere tutte le proprietà il cui dominio è un \textbf{<\#ComponentModule>}.
\end{info}

	\item \textbf{<\#ownsComponentHw>}: Specializzazione di \textit{<\#hasComponentHw>} per modellare il fatto che un <#ComponentModule> possiede un \textit{<\#ComponentHw>};

	\item \boldYellow{<\#isOwnedComponentHw>}: Proprietà inversa di \textit{<\#ownsComponentHw>};

	\item \textbf{<\#ownsSensorHw>}: Specializzazione di \textit{<\#ownsComponentHw>} e di \textit{<\#hasSensorHw>} utilizzata per modellare il fatto che un <#ComponentModule> possiede un \textit{<\#SensorHw>};

	\item \boldYellow{<\#isOwnedSensorHw>}: Proprietà inversa di \textit{<\#ownsSensorHw>};

	\item \textbf{<\#ownsActuatorHw>}: Specializzazione di \textit{<\#ownsComponentHw>} e di \textit{<\#hasActuatorHw>} utilizzata per modellare il fatto che un <#ComponentModule> possiede un \textit{<\#ActuatorHw>};

	\item \boldYellow{<\#isOwnedActuatorHw>}: Proprietà inversa di \textit{<\#ownsActuatorHw>};

	\item \textbf{<\#ownsDigitalLightHw>}: Specializzazione di \textit{<\#ownsActuatorHw>} e di \textit{<\#hasDigitalLightHw>} utilizzata per modellare il fatto che un <#ComponentModule> possiede un \textit{<\#DigitalLightHw>};

	\item \boldYellow{<\#isOwnedDigitalLightHw>}: Proprietà inversa di \textit{<\#ownsDigitalLightHw>};

	\item \textbf{<\#hasHandlerOutputType>}: Proprietà utilizzata per definire il tipo di output fornito da un \textit{<\#OperationHandler>};

	\item \textbf{<\#hasHandlerInputSchema>}: Proprietà utilizzata per definire quali tipi di dati possono essere forniti in input ad un \textit{<\#OperationHandler>};

	\item \textbf{<\#hasThingDescriptor>}: Proprietà utilizzata per legare un \textit{<\#Edge>} al suo Thing Descriptor (\textit{td:Thing}).
\end{itemize}

\subsection{Individui}
Sono stati realizzati degli individui di esempio.

\noindent Nella serra che è stata creata (\textit{<\#angelo-greenhouse>}) è presente un solo ambiente (\textit{<\#angelo-environment>}) che contiene a sua volta una sola pianta (\textit{<\#angelo-basil>}).

\noindent Il dispositivo Edge che è stato creato (\textit{<\#angelo-edge>}) è composto da un solo modulo che gestisce l'illuminazione (\textit{<\#digital-light-module>}). Quest'ultimo è composto di tre Operation Handler: uno di tipo \textit{<\#IsOnDigitalLightHandler>}, uno di tipo \textit{<\#TurnOnDigitalLightHandler>} ed uno di tipo \textit{<\#TurnOffDigitalLightHandler>}

\noindent Il modulo contiene due componenti di tipo <\#DigitalLightHw>, che sono proprio i componenti su cui possono interagire i tre Operation Handler appena nominati.

\noindent Infine il dispositivo edge possiede un Thing Descriptor le cui ActionAffordance e PropertyAffordance sono mappate con quelle di tutti gli Operation Handler di tutti i Moduli che possiede.
