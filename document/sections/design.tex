\section{Design}

\centeredImage{img/brittany-rdf-graph.png}{Diagramma sommario delle classi e delle proprietà (Struttura Principale)}{0.8}

\noindent Il dominio è stato modellato utilizzando l'Ubiquitous Language appena illustrato. Sono stati inoltre ovviamente aggiunti elementi più di dettaglio e di basso livello, come le classi e le proprietà che fanno parte delle gerarchie "is a", più specifiche e per ora ancora mai nominate.

\noindent Questo grafico illustra solo le classi e le relazioni che fanno parte della struttura principale. In seguito vengono illustrate tutte le classi e tutte le proprietà. Le relazioni in giallo rappresentano le proprietà inverse di quelle in nero che hanno la freccia nel verso opposto.

\subsection{Concetti (Classi)}
Seguono gli identificatori delle classi create. Vengono riportate alcune note importanti. Dettagli più accurati ed approfonditi sono riportati nel file RDF \textit{brittany.ttl}:

\begin{itemize}
	\item \textbf{<\#NameRestriction>}:Classe di tipo \textit{owl:Restriction}, non anonima perché riutilizzata da più classi;

	\item \textbf{<\#Greenhouse>}:Utilizzato per modellare il concetto di Serra;

	\item \textbf{<\#Monitorable>}: Utilizzato per modellare il concetto di un "qualcosa" che può essere controllato da una classe di tipo \textit{<\#Edge>}. \textit{<\#Environment>} e \textit{<\#Plant>} sono sottoclassi di questa classe;

	\item \textbf{<\#Environment>}: Utilizzato per modellare il concetto di Ambiente;

	\item \textbf{<\#Plant>}: Utilizzato per modellare il concetto di Pianta;

	\item \textbf{<\#Edge>}: Utilizzato per modellare il concetto di Edge;

	\item \textbf{<\#Module>}: Utilizzato per modellare il concetto di Modulo;

	\item \textbf{<\#ComponentModule>}: Si tratta di una specializzazione di \textit{<\#Module>} che agisce necessariamente su dei componenti hardware (\textit{<\#ComponentHw>});

	\item \textbf{<\#DigitalLightModule>}: Una specializzazione di \textit{<\#ComponentModule>} che si occupa del sistema di illuminazione;

	\item \textbf{<\#OperationHandler>}: Utilizzato per modellare il concetto di Operation Handler;

	\item \textbf{<\#ComponentOperationHandler>}: Specializzazione di \textit{<\#OperationHandler>} che agisce necessariamente su dei componenti hardware (\textit{<\#ComponentHw>});

	\item \textbf{<\#ActionOperationHandler>}: Specializzazione di \textit{<\#OperationHandler>} utilizzata per definire un'operazione che effettua un'azione;

	\item \textbf{<\#PropertyOperationHandler>}: Specializzazione di \textit{<\#OperationHandler>} utilizzata per definire un'operazione che ricava una proprietà;

	\item \textbf{<\#DigitalLightHandler>}: Specializzazione di \textit{<\#ComponentOperationHandler>} per definire operazioni legate al sistema di illuminazione;

	\item \textbf{<\#IsOnDigitalLightHandler>}: Specializzazione di \textit{<\#DigitalLightHandler>} e di \textit{<\#PropertyOperationHandler>}. L'operazione si occupa di verificare se un determinato componente che fa parte del sistema di illuminazione è acceso.

	\item \textbf{<\#TurnOnDigitalLightHandler>}: Specializzazione di \textit{<\#DigitalLightHandler>} e di \textit{<\#PropertyOperationHandler>}. L'operazione si occupa di accendere un determinato componente che fa parte del sistema di illuminazione;

	\item \textbf{<\#TurnOffDigitalLightHandler>}: Specializzazione di \textit{<\#DigitalLightHandler>} e di \textit{<\#PropertyOperationHandler>}. L'operazione si occupa di spegnere un determinato componente che fa parte del sistema di illuminazione;

	\item \textbf{<\#ComponentHw>}: Utilizzato per modellare il concetto di Componente Hardware;

	\item \textbf{<\#SensorHw>}: Specializzazione di \textit{<\#ComponentHw>}. Un componente che ricava dati dall'area circostante;

	\item \textbf{<\#ActuatorHw>}: Specializzazione di \textit{<\#ComponentHw>}. Un componente che agisce sull'area circostante;

	\item \textbf{<\#DigitalLightHw>}: Specializzazione di \textit{<\#ActuatorHw>}. Un attuatore che si occupa del sistema di illuminazione;

	\item \textbf{<\#BrittanyThingDescriptor>}: Specializzazione di \textit{td:Thing}. Un Thing Descriptor che ha alcune proprietà sempre dello stesso tipo. Brittany utilizza infatti sempre alcuni tipi di valori e questa classe permette di porre quei vincoli.

	\item \textbf{<\#BrittanyForm>}: Specializzazione di \textit{td:Form}. Un Form che ha ha alcune proprietà sempre dello stesso tipo.

\end{itemize}

\subsection{Relazioni (Proprietà)}
Seguono gli identificatori delle proprietà create. Vengono riportate alcune note importanti. Dettagli più accurati ed approfonditi sono riportati nel file RDF \textit{brittany.ttl}.

\definecolor{yellow(munsell)}{rgb}{0.65, 0.50, 0.0}
\newcommand\boldYellow[1]{\textcolor{yellow(munsell)}{\textbf{#1}}}

\subsubsection{ObjectProperty}
Vengono ora elencate le ObjectProperty, ossia le proprietà il cui range deve essere necessariamente un individuo.
\begin{itemize}
	\item \textbf{<\#hasEnvironment>}: Proprietà utilizzata per modellare il fatto all'interno di \textit{<\#Greenhouse>} c'è un \textit{<\#Environment>};

	\item \boldYellow{<\#isEnvironmentOf>}: Proprietà inversa di \textit{<\#hasEnvironments>};

	\item \textbf{<\#hasPlant>}: Proprietà utilizzata per modellare il fatto che in uno spazio è presente una  \textit{<\#Plant>};

	\item \boldYellow{<\#isPlantIn>}: Proprietà inversa di \textit{<\#hasPlant>};

	\item \textbf{<\#hasEnvironmentPlant>}: Specializzazione di \textit{<\#hasPlant>}, utilizzata per modellare il fatto che in un \textit{<\#Environment>} è presente una  \textit{<\#Plant>};

	\item \boldYellow{<\#isPlantInEnvironment>}: Proprietà inversa di \textit{<\#hasEnvironmentPlant>};

	\item \textbf{<\#hasGreenhousePlant>}: Specializzazione di \textit{<\#hasPlant>}, utilizzata per modellare il fatto che in una \textit{<\#Greenhouse>} è presente una  \textit{<\#Plant>};

	\item \boldYellow{<\#isPlantInGreenhouse>}: Proprietà inversa di \textit{<\#hasGreenhousePlant>};

	\item \textbf{<\#hasGreenhouseEdge>}: Proprietà utilizzata per modellare il fatto che in una \textit{<\#Greenhouse>} è presente un \textit{<\#Edge>};

	\item \boldYellow{<\#isEdgeInGreenhouse>}: Proprietà inversa di \textit{<\#hasGreenhouseEdge>};

	\item \textbf{<\#hasModule>}: Proprietà utilizzata per modellare il fatto che è presente un \textit{<\#Module>} all'interno di un \textit{<\#Edge>};

	\item \boldYellow{<\#isModuleIn>}: Proprietà inversa di \textit{<\#hasModule>};

	\item \textbf{<\#hasHandler>}: Proprietà utilizzata per modellare il fatto che esiste un \textit{<\#OperationHandler>} all'interno di un \textit{<\#Module>};

	\item \boldYellow{<\#isHandlerIn>}: Proprietà inversa di \textit{<\#hasHandler>};

	\item \textbf{<\#monitor>}: Proprietà utilizzata per modellare il fatto che un \textit{<\#Edge>} può essere posto in un qualsiasi \textit{<\#Monitorable>};

	\item \boldYellow{<\#isMonitored>}: Proprietà inversa di \textit{<\#monitor>};

	\item \textbf{<\#monitorPlant>}: Specializzazione di \textit{<\#monitor>} dedicata al monitor di un \textit{<\#Plant>};

	\item \boldYellow{<\#isPlantMonitored>}: Proprietà inversa di \textit{<\#monitorPlant>};

	\item \textbf{<\#monitorEnvironment>}: Specializzazione di \textit{<\#monitor>} dedicata al monitor di una \textit{<\#Environment>};

	\item \boldYellow{<\#isEnvironmentMonitored>}: Proprietà inversa di \textit{<\#monitorEnvironment>};

	\item \textbf{<\#hasComponentHw>}: Proprietà utilizzata per modellare il fatto che qualcosa possiede un \textit{<\#ComponentHw>};

	\item \boldYellow{<\#isComponentHwIn>}: Proprietà inversa di \textit{<\#hasComponentHw>};

	\item \textbf{<\#hasSensorHw>}: Specializzazione di \textit{<\#hasComponentHw>} utilizzata per modellare il fatto che qualcosa possiede un \textit{<\#SensorHw>};

	\item \boldYellow{<\#isSensorHwIn>}: Proprietà inversa di \textit{<\#hasSensorHw>};

	\item \textbf{<\#hasActuatorHw>}: Specializzazione di \textit{<\#hasComponentHw>} utilizzata per modellare il fatto che qualcosa possiede un \textit{<\#ActuatorHw>};

	\item \boldYellow{<\#isActuatorHwIn>}: Proprietà inversa di \textit{<\#hasActuatorHw>};

	\item \textbf{<\#hasDigitalLightHw>}: Specializzazione di \textit{<\#hasActuatorHw>} utilizzata per modellare il fatto che qualcosa possiede un \textit{<\#DigitalLightHw>};

	\item \boldYellow{<\#isDigitalLightHwIn>}: Proprietà inversa di \textit{<\#hasDigitalLightHw>};

\begin{info}[owns:]
	La parola chiave "owns" è stata utilizzata per contraddistinguere tutte le proprietà il cui dominio è un \textbf{<\#ComponentModule>}.
\end{info}

	\item \textbf{<\#ownsComponentHw>}: Specializzazione di \textit{<\#hasComponentHw>} per modellare il fatto che un <\#ComponentModule> possiede un \textit{<\#ComponentHw>};

	\item \boldYellow{<\#isOwnedComponentHw>}: Proprietà inversa di \textit{<\#ownsComponentHw>};

	\item \textbf{<\#ownsSensorHw>}: Specializzazione di \textit{<\#ownsComponentHw>} e di \textit{<\#hasSensorHw>} utilizzata per modellare il fatto che un <\#ComponentModule> possiede un \textit{<\#SensorHw>};

	\item \boldYellow{<\#isOwnedSensorHw>}: Proprietà inversa di \textit{<\#ownsSensorHw>};

	\item \textbf{<\#ownsActuatorHw>}: Specializzazione di \textit{<\#ownsComponentHw>} e di \textit{<\#hasActuatorHw>} utilizzata per modellare il fatto che un <\#ComponentModule> possiede un \textit{<\#ActuatorHw>};

	\item \boldYellow{<\#isOwnedActuatorHw>}: Proprietà inversa di \textit{<\#ownsActuatorHw>};

	\item \textbf{<\#ownsDigitalLightHw>}: Specializzazione di \textit{<\#ownsActuatorHw>} e di \textit{<\#hasDigitalLightHw>} utilizzata per modellare il fatto che un <\#ComponentModule> possiede un \textit{<\#DigitalLightHw>};

	\item \boldYellow{<\#isOwnedDigitalLightHw>}: Proprietà inversa di \textit{<\#ownsDigitalLightHw>};

\begin{info}[collect:]
	La parola chiave "collect" è stata utilizzata per contraddistinguere tutte le proprietà il cui dominio è un \textbf{<\#Edge>}.
\end{info}

	\item \textbf{<\#collectComponentHw>}: Specializzazione di \textit{<\#hasComponentHw>} utilizzata per modellare il fatto che un <\#Edge> possiede un \textit{<\#ComponentHw>};

	\item \boldYellow{<\#isCollectedComponentHw>}: Proprietà inversa di \textit{<\#collectComponentHw>};

	\item \textbf{<\#collectSensorHw>}: Specializzazione di \textit{<\#collectComponentHw>} e di \textit{<\#hasSensorHw>} utilizzata per modellare il fatto che un <\#Edge> possiede un \textit{<\#SensorHw>};

	\item \boldYellow{<\#isCollectedSensorHw>}: Proprietà inversa di \textit{<\#collectSensorHw>};

	\item \textbf{<\#collectActuatorHw>}: Specializzazione di \textit{<\#collectComponentHw>} e di \textit{<\#hasActuatorHw>} utilizzata per modellare il fatto che un <\#Edge> possiede un \textit{<\#ActuatorHw>};

	\item \boldYellow{<\#isCollectedActuatorHw>}: Proprietà inversa di \textit{<\#collectActuatorHw>};

	\item \textbf{<\#collectDigitalLightHw>}: Specializzazione di \textit{<\#collectActuatorHw>} e di \textit{<\#hasDigitalLightHw>} utilizzata per modellare il fatto che un <\#Edge> possiede un \textit{<\#DigitalLightHw>};

	\item \boldYellow{<\#isCollectedDigitalLightHw>}: Proprietà inversa di \textit{<\#collectDigitalLightHw>};

\begin{info}[handle:]
	La parola chiave "handle" è stata utilizzata per contraddistinguere tutte le proprietà il cui dominio è un \textbf{<\#ComponentOperationHandler>}.
\end{info}

	\item \textbf{<\#handleComponentHw>}: Specializzazione di \textit{<\#hasComponentHw>} utilizzata per modellare il fatto che un <\#ComponentOperationHandler> agisce su un \textit{<\#ComponentHw>};

	\item \boldYellow{<\#isHandledComponentHw>}: Proprietà inversa di \textit{<\#handleComponentHw>};

	\item \textbf{<\#handleSensorHw>}: Specializzazione di \textit{<\#handleComponentHw>} e di \textit{<\#hasSensorHw>} utilizzata per modellare il fatto che un <\#ComponentOperationHandler> agisce su un \textit{<\#SensorHw>};

	\item \boldYellow{<\#isHandledSensorHw>}: Proprietà inversa di \textit{<\#handleSensorHw>};

	\item \textbf{<\#handleActuatorHw>}: Specializzazione di \textit{<\#handleComponentHw>} e di \textit{<\#hasActuatorHw>} utilizzata per modellare il fatto che un <\#ComponentOperationHandler> agisce su \textit{<\#ActuatorHw>};

	\item \boldYellow{<\#isHandledActuatorHw>}: Proprietà inversa di \textit{<\#handleActuatorHw>};

	\item \textbf{<\#handleDigitalLightHw>}: Specializzazione di \textit{<\#handleActuatorHw>} e di \textit{<\#hasDigitalLightHw>} utilizzata per modellare il fatto che un <\#ComponentOperationHandler> agisce su un \textit{<\#DigitalLightHw>};

	\item \boldYellow{<\#isHandledDigitalLightHw>}: Proprietà inversa di \textit{<\#handleDigitalLightHw>};

	\item \textbf{<\#hasHandlerInteractionAffordance>}:  Proprietà utilizzata per modellare il fatto che un <\#OperationHandler> è collegato ad una Property Affordance \textit{(td:PropertyAffordance)};

	\item \boldYellow{<\#isInteractionAffordanceOfHandler>}: Proprietà inversa di \textit{<\#hasHandlerInteractionAffordance>};

	\item \textbf{<\#hasHandlerPropertyAffordance>}: Specializzazione di \textit{<\#hasHandlerInteractionAffordance>} utilizzata per modellare il fatto che un <\#OperationHandler> è collegato ad una Property Affordance \textit{(td:PropertyAffordance)};

	\item \boldYellow{<\#isPropertyAffordanceOfHandler>}: Proprietà inversa di \textit{<\#hasHandlerPropertyAffordance>};

	\item \textbf{<\#hasHandlerActionAffordance>}: Specializzazione di \textit{<\#hasHandlerInteractionAffordance>} utilizzata per modellare il fatto che un <\#OperationHandler> è collegato ad una Action Affordance \textit{(td:ActionAffordance)};

	\item \boldYellow{<\#isActionAffordanceOfHandler>}: Proprietà inversa di \textit{<\#hasHandlerActionAffordance>};

	\item \textbf{<\#hasHandlerInputSchema>}: Proprietà utilizzata per definire quali tipi di dati possono essere forniti in input ad un \textit{<\#OperationHandler>};

	\item \boldYellow{<\#isInputSchemaOfHandler>}: Proprietà inversa di \textit{<\#hasHandlerInputSchema>};

	\item \textbf{<\#hasThingDescriptor>}: Proprietà utilizzata per legare un \textit{<\#Edge>} al suo Thing Descriptor (\textit{<\#BrittanyThingDescriptor>});

	\item \boldYellow{<\#isThingDescriptor>}: Proprietà inversa di \textit{<\#hasThingDescriptor>};

	\item \boldYellow{<\#isInteractionAffordanceOf>}: Proprietà inversa di \textit{td:hasInteractionAffordance};

	\item \boldYellow{<\#isPropertyAffordanceOf>}: Proprietà inversa di \textit{td:hasPropertyAffordance};

	\item \boldYellow{<\#isActionAffordanceOf>}: Proprietà inversa di \textit{td:hasActionAffordance}.

\end{itemize}

\subsubsection{DatatyeProperty}
\begin{itemize}
	\item \textbf{foaf:name}: Proprietà utilizzata per definire il nome di risorse;

	\item \textbf{wot:href}: Superproprietà di \textit{hctl:hasTarget} utilizzata per specificare degli URL;

	\item \textbf{hctl:hasTarget}: Proprità utilizzata per definite un path all'interno di un Form del Thing Descriptor;

	\item \textbf{hctl:forContentType}: Proprietà utilizzata per definire in che modo i dati vengono scambiati nelle Interaction Affordance \textit{(td:InteractionAffordance)};

	\item \textbf{<\#hasHandlerOutputType>}: Proprietà utilizzata per definire il tipo di output fornito da un \textit{<\#OperationHandler>}.
\end{itemize}

\subsection{Individui}
Sono stati realizzati degli individui di esempio:\newline

\noindent Sono stati creati innanzitutto due individui di tipo \textit{foaf:Person}, \textit{<\#angelo-filaseta>} e \textit{<\#giorgia-savo>}.\newline

\noindent Per ciascuno degli individui è stata creata una serra, rispettivamente \textit{<\#angelo-greehouse>} e \textit{<\#giorgia-greenhouse>}. La creazione degli individui di tipo persona e il legame alle serre attraverso la proprietà \textit{dbo:owner} non è necessaria ma è stata creata per rendere più chiaro il dominio.\newline

\noindent Nella serra (\textit{<\#angelo-greenhouse>}) sono presenti due ambienti (\textit{<\#angelo-environment-one>} , \textit{<\#angelo-environment-two>}) che a loro volta contengono rispettivamente le piante \{\textit{<\#angelo-basil>}, \textit{<\#angelo-tomato>}, \textit{<\#angelo-parsley>}\} e \{\textit{<\#angelo-onion>}, \textit{<\#angelo-lettuce>})\}.\newline

\noindent Nella serra (\textit{<\#giorgia-greenhouse>}) è presente un solo ambiente (\textit{<\#giorgia-environment>}) che contiene le piante \{\textit{<\#giorgia-cucumber>, <\#giorgia-potato>})\}. \newline

\noindent Entrambe le serre contengono un dispositivo Edge, rispettivamente \textit{<\#angelo-edge>} e \textit{<\#giorgia-edge>}. Entrambi gli Edge contengono un solo modulo di tipo \textit{<\#DigitalLightModule>}. La configurazione è la stessa:
\begin{enumerate}
	\item Il nome dell'individuo che rappresenta il modulo è \textit{<\#NOME-edge-module-digital-light>};
	\item Ognuno ha a disposizione i tre handler \textit{<\#NOME-handler-turn-on-digital-light>}, \textit{<\#NOME-handler-turn-off-digital-light>}, \textit{<\#NOME-handler-is-on-digital-light>};
\end{enumerate}
La differenza è che gli handler \textit{<\#angelo-..>} hanno due individui di tipo \textit{<\#DigitalLightHw>} collegati (\textit{<\#angelo-digital-light-sx>} e \textit{<\#angelo-digital-light-sx>}) mentre quelli \textit{<\#giorgia-..>} ne hanno solo uno (\textit{<\#giorgia-digital-light-center>}).\newline

\noindent Infine entrambi gli individui di tipo \textit{<\#Edge>} sono collegati ad un individuo di tipo \textit{<\#BrittanyThingDescriptor>}, rispettivamente \textit{<\#angelo-td-mock-ip>} e \textit{<\#giorgia-td-mock-ip>} le cui ActionAffordance e PropertyAffordance sono mappate con quelle di tutti gli Operation Handler di tutti i Moduli posseduti.

\subsection{Restrizioni}
\subsubsection{Cardinalità}
Vengono ora elencate tutte le restrizioni sulle cardinalità presenti in Brittany RDF:
\begin{itemize}
	\item \textbf{<\#Greenhouse>}:
	\begin{itemize}
		\item Esattamente un \textit{foaf:name};
		\item Almeno un \textit{<\#Environment>} legato attraverso la proprietà \textit{<\#hasEnvironment>};
	\end{itemize}
	\item \textbf{<\#Edge>}:
	\begin{itemize}
		\item Esattamente un \textit{<\#BrittanyThingDescriptor>} legato attraverso la proprietà \textit{<\#hasThingDescriptor>};
		\item Almeno un \textit{<\#Module>} legato attraverso la proprietà \textit{<\#hasModule>};
	\end{itemize}
	\item \textbf{<\#Module>}:
	\begin{itemize}
		\item Esattamente un \textit{foaf:name};
		\item Almeno un \textit{<\#OperationHandler>} legato attraverso la proprietà \textit{<\#hasHandler>};
	\end{itemize}
	\item \textbf{<\#OperationHandler>}:
	\begin{itemize}
		\item Esattamente un \textit{foaf:name};
	\end{itemize}
	\item \textbf{<\#ComponentHw>}:
	\begin{itemize}
		\item Esattamente un \textit{foaf:name}.
	\end{itemize}
\end{itemize}
\subsubsection{Valori}
Vengono ora elencate tutte le restrizioni sui valori presenti in Brittany RDF:
\begin{itemize}
	\item \textbf{<\#DigitalLightModule>}:
	\begin{itemize}
		\item Ogni individuo ha una proprietà \textit{foaf:name} con valore \textit{"digital-light"};
	\end{itemize}
	\item \textbf{<\#IsOnDigitalLightHandler>}:
		\begin{itemize}
		\item Ogni individuo ha una proprietà \textit{foaf:name} con valore \textit{"isOn"};
		\item Ogni individuo ha una proprietà \textit{<\#hasHandlerOutputType>} con valore \textit{"boolean"};
	\end{itemize}
	\item \textbf{<\#TurnOnDigitalLightHandler>}:
	\begin{itemize}
		\item Ogni individuo ha una proprietà \textit{foaf:name} con valore \textit{"turnOn"};
		\item Ogni individuo ha una proprietà \textit{<\#hasHandlerOutputType>} con valore \textit{"string"};
	\end{itemize}
	\item \textbf{<\#TurnOffDigitalLightHandler>}:
	\begin{itemize}
		\item Ogni individuo ha una proprietà \textit{foaf:name} con valore \textit{"turnOff"};
		\item Ogni individuo ha una proprietà \textit{<\#hasHandlerOutputType>} con valore \textit{"string"};
	\end{itemize}
	\item \textbf{<\#BrittanyThingDescriptor>}:
	\begin{itemize}
		\item Ogni individuo ha una proprietà \textit{td:definesSecurityScheme} con valore \textit{<\#nosec-schema>};
		\item Ogni individuo ha una proprietà \textit{td:hasSecurityConfiguration} con valore \textit{<\#no-sec>};
	\end{itemize}
	\item \textbf{<\#BrittanyForm>}:
	\begin{itemize}
		\item Ogni individuo ha una proprietà \textit{hctl:forContentType} con valore \textit{"application/json"}.
	\end{itemize}
\end{itemize}
\subsection{Osservazioni aggiuntive}
\subsubsection{Design del Thing Descriptor}
Il Thing Descriptor è stato pensato per essere utilizzato principalmente in formato JSON-LD. Il funzionamento non subisce nessun tipo di impatto se cambia anche il tipo di codifica, in quanto è sempre possibile un'operazione di conversione. È però fondamentale che la struttura del Thing Descriptor rimanga invariata. I nomi delle proprietà utilizzate sono mantenute nella W3 Thing Descriptor Ontology~\cite{ThingDes54:online}.\newline
\noindent Esistono tuttavia degli elementi che non è necessario aggiungere alla nuova implementazioni di Thing Descriptor. Rispetto alla versione illustrata nella sottosezione \ref{tdform} non è più necessario aggiungere il campo \textit{"modules"}, ed è anche possibile rimuovere tutti i campi "module" in ogni Interaction Affordance.\newline
Utilizzando RDF è ora possibile risalire a queste informazioni a partire dalle informazioni già contenute nel TD.
\begin{warn}
	È però necessario assumere che nella versione finale ed integrata del sistema Brittany questi dati saranno accessibili e utilizzabili in qualche modo dal microservizio \textit{"Greenhouse Core"}.
\end{warn}
