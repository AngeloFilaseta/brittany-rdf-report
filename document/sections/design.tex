\section{Design}

\subsection{Ubiquitous Language}
Viene di seguito illustrato come è stato definito l’Ubiquitous Language utilizzato nel sistema. Viene riportato il solo sottoinsieme di componenti di interesse per il progetto RDF. Verranno riportati in questa relazione solo alcuni dei concetti significativi, in quanto RDF è la tecnologia più adatta a modellare e definire l'Ubiquitous Language, ed è quindi possibile approfondire visionando il documento \textit{brittany.ttl}. Tra parentesi viene riportato il nome in italiano.

\begin{itemize}
	\item \textbf{Greenhouse (Serra)}: Una struttura artificiale suddiviso in \textit{ambienti} costruita appositamente per coltivare \textit{piante}.
	\item \textbf{Environment (Ambiente)}: Sezione di una \textit{serra} che contiene più \textit{piante}. Una \textit{serra} è composta da almeno un \textit{ambiente}.
	\item \textbf{Plant (Pianta)}: Unità vegetale contenuta all’interno di un \textit{ambiente} della \textit{serra}.
	\item \textbf{Edge (Edge)}: Una macchina elettronica che colleziona dati da \textit{sensori} e che può manipolare degli \textit{attuatori}. Un \textit{Edge} è composto da \textit{moduli}. Un \textit{Edge} può essere posizionato in un \textit{ambiente} o su una \textit{pianta}.
	\item \textbf{Module (Module)}: Un componente logico che può essere aggiunto ad un \textit{Edge} ed è composto da più \textit{Operation Handler}.
	\item \textbf{Operation Handler (Operation Handler)}: Un'operazione che utilizza degli argomenti per della computazione e ritorna un risultato.
	\item \textbf{Hardware Component (Componente Hardware)}: Un Componente Hardware che può essere utilizzato da un Operation Handler.
\end{itemize}

\subsection{RDF e RDFS}

\centeredImage{img/brittany-rdf-graph.png}{Diagramma sommario delle classi e delle proprietà}{0.8}

\noindent Il dominio è stato modellato utilizzando l'Ubiquitous Language appena illustrato. Sono stati inoltre ovviamente aggiunti elementi più di dettaglio e di basso livello, come le classi e le proprietà che fanno parte delle gerarchie "is a", più specifiche e per ora ancora mai nominate.
\subsubsection{Concetti (Classi)}
Seguono gli identificatori delle classi utilizzate. Vengono riportate alcune note importanti. Dettagli più accurati ed approfonditi sono riportati nel file RDF \textit{brittany.ttl}.

\paragraph{\#Greenhouse}
\paragraph{\#Monitorable}
\paragraph{\#Environment}
\paragraph{\#Plant}
\paragraph{\#Edge}
\paragraph{\#Module}
\paragraph{\#ComponentModule}
\paragraph{\#DigitalLightModule}
\paragraph{\#OperationHandler}
\paragraph{\#ComponentOperationHandler}
\paragraph{\#ActionOperationHandler}
\paragraph{\#PropertyOperationHandler}
\paragraph{\#DigitalLightHandler}
\paragraph{\#IsOnDigitalLightHandler}
\paragraph{\#TurnOnDigitalLightHandler}
\paragraph{\#TurnOffDigitalLightHandler}
\paragraph{\#ComponentHw}
\paragraph{\#SensorHw}
\paragraph{\#ActuatorHw}
\paragraph{\#DigitalLightHw}

\subsubsection{Relazioni (Proprietà)}

\subsubsection{Individui}

\subsection{Riuso di ontologie esistenti}
