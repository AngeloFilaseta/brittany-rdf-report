\section{Descrizione del dominio di "Brittany"}

\subsection{Scope del progetto}

Brittany è un sistema di serre il cui scopo è automatizzare il più possibile i processi di crescita di alcune piante.

\noindent Il sistema è composto da dispositivi edge che monitorano la situazione delle piante e dell'ambiente circostante (tramite sensori) ed eventualmente vi agiscono (tramite attuatori). Ogni edge dispone di funzionalità che dipendono dall'hardware a cui essi sono collegati.
I dati che vengono campionati vengono raccolti in una base di dati. Gli utenti possono configurare le serre in base alle loro esigenze, visionare la situazione in un qualsiasi momento e possono essere avvisati in caso di situazioni anomale~\cite{Brittany65:online}.

\subsection{Descrizione ad alto livello del sistema}
Brittany è un sistema suddiviso in micro servizi:
\begin{itemize}
	\item \textbf{Edge}: Si tratta del componente che manipola i sensori e gli attuatori all’interno degli ambienti della serra. In un ambiente possono esserci più Edge;
	\item \textbf{Greenhouse Core}: Si tratta di un ambiente in cui degli agenti hanno un loro lifecycle, agiscono sugli Edge e contattano i Service qualora necessario.
	\item \textbf{Auth Service}: Si tratta del servizio di autenticazione, che permette di effettuare le operazioni di registrazione e login al sistema.
	\item \textbf{Settings Service}: Il servizio che consente di creare, modificare o eliminare determinate impostazioni utilizzate all’interno di Grennhouse Core, come ad esempio ogni quanto tempo campionare determinati tipi di dato.
	\item \textbf{Persistence Service}: Il servizio che permette di salvare e recuperare dati campionati.
	\item \textbf{Web Client}: Il servizio che offre possibilità ad un client di interagire con il sistema.
\end{itemize}
Greenhouse core deve necessariamente essere collocato nella stessa rete LAN in cui si trovano gli apparecchi di tipo Edge.

\centeredImage{img/architecture.png}{Architettura del sistema "Brittany"}{0.7}

\subsubsection{Parte del dominio presa in considerazione per Brittany RDF}
\noindent La parte del dominio che è interessante da rappresentare e formalizzare è quella inerente ai dispositivi di tipo "Edge". Si tratta della parte del sistema più Core e dunque più ricca di concetti. \newline
Anche i parametri di configurazione di cui si occupa "Settings Service" sarebbero stati interessanti, ma sono già stati modellati accuratamente utilizzando altre tecnologie e sono stati quindi per ora esclusi.

\centeredImage{img/represent.png}{Parte del sistema "Brittany" presa in considerazione per il progetto corrente}{0.7}

\subsection{Ubiquitous Language}
Viene di seguito illustrato come è stato definito l’Ubiquitous Language utilizzato nel sistema.

\noindent Viene riportato il solo sottoinsieme di componenti di interesse per il progetto RDF. Verranno riportati in questa relazione solo alcuni dei concetti significativi, in quanto RDF è la tecnologia più adatta a modellare e definire l'Ubiquitous Language, ed è quindi possibile approfondire visionando il documento \textit{brittany.ttl}. Tra parentesi viene riportato il nome in italiano.

\begin{itemize}
	\item \textbf{Greenhouse (Serra)}: Una struttura artificiale suddiviso in \textit{ambienti} costruita appositamente per coltivare \textit{piante}.
	\item \textbf{Environment (Ambiente)}: Sezione di una \textit{serra} che contiene più \textit{piante}. Una \textit{serra} è composta da almeno un \textit{ambiente}.
	\item \textbf{Plant (Pianta)}: Unità vegetale contenuta all’interno di un \textit{ambiente} della \textit{serra}.
	\item \textbf{Edge (Edge)}: Una macchina elettronica che colleziona dati da \textit{sensori} e che può manipolare degli \textit{attuatori}. Un \textit{Edge} è composto da \textit{moduli}. Un \textit{Edge} può essere posizionato in un \textit{ambiente} o su una \textit{pianta}.
	\item \textbf{Module (Module)}: Un componente logico che può essere aggiunto ad un \textit{Edge} ed è composto da più \textit{Operation Handler}.
	\item \textbf{Operation Handler (Operation Handler)}: Un'operazione che utilizza degli argomenti per della computazione e ritorna un risultato.
	\item \textbf{Hardware Component (Componente Hardware)}: Un Componente Hardware che può essere utilizzato da un Operation Handler.
\end{itemize}
