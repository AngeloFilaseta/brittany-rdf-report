\section{Descrizione del dominio}

\subsection{Descrizione ad alto livello}

\subsection{Requisiti}

\subsubsection{Requisiti Funzionali}

I requisiti funzionali riguardano le funzionalità che il sistema deve mettere a disposizione all’utente.

\noindent Il principale requisito funzionale consiste nel rappresentare la struttura del dominio del software "Brittany" e le varie relazioni tra i componenti, che infatti non sono mantenute in nessun modo. Una definizione rigorosa permetterebbe una più semplice gestione e catalogazione dei componenti, in particolare la rappresentazione vorrebbe permettere di:
\begin{enumerate}
	\item creare, modificare, cancellare una serra;
	\item creare, aggiungere e rimuovere ambienti ad una serra;
	\item aggiungere e rimuovere piante da ambienti;
	\item personalizzare dispositivi denominati "edge" in modo che possano monitorare ed effettuare azioni su una pianta o un ambiente;
	\item creare e definire i comportamenti dei moduli che potranno essere utilizzati per personalizzare gli edge.
	\item definire quali operazioni può eseguire ciascun modulo e su quali componenti hardware all'interno della serra.
\end{enumerate}

\subsubsection{Requisiti Implementativi}

I requisiti di implementazione vincolano l’intera fase di realizzazione del sistema, ad esempio richiedendo l’uso di uno specifico linguaggio di programmazione e/o di uno specifico tool software.

\noindent Per lo sviluppo di questo progetto si è vincolati ad utilizzare le tecnologie \textbf{RDF} e \textbf{RDF Schema}.

\subsection{Ubiquitous Language}
Viene di seguito illustrato come è stato definito l’Ubiquitous Language utilizzato nel sistema. Viene riportato il solo sottoinsieme di componenti di interesse per il progetto RDF. Verranno riportati in questa relazione solo alcuni dei concetti significativi, in quanto RDF è la tecnologia più adatta a modellare e definire l'Ubiquitous Language, ed è quindi possibile approfondire visionando il documento \textit{brittany.ttl}. Tra parentesi viene riportato il nome in italiano.

\begin{itemize}
	\item \textbf{Greenhouse (Serra)}: Una struttura artificiale suddiviso in \textit{ambienti} costruita appositamente per coltivare \textit{piante}.
	\item \textbf{Environment (Ambiente)}: Sezione di una \textit{serra} che contiene più \textit{piante}. Una \textit{serra} è composta da almeno un \textit{ambiente}.
	\item \textbf{Plant (Pianta)}: Unità vegetale contenuta all’interno di un \textit{ambiente} della \textit{serra}.
	\item \textbf{Edge (Edge)}: Una macchina elettronica che colleziona dati da \textit{sensori} e che può manipolare degli \textit{attuatori}. Un \textit{Edge} è composto da \textit{moduli}. Un \textit{Edge} può essere posizionato in un \textit{ambiente} o su una \textit{pianta}.
	\item \textbf{Module (Module)}: Un componente logico che può essere aggiunto ad un \textit{Edge} ed è composto da più \textit{Operation Handler}.
	\item \textbf{Operation Handler (Operation Handler)}: Un'operazione che utilizza degli argomenti per della computazione e ritorna un risultato.
	\item \textbf{Hardware Component (Componente Hardware)}: Un Componente Hardware che può essere utilizzato da un Operation Handler.
\end{itemize}
