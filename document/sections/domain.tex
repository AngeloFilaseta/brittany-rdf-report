\section{Descrizione del dominio di "Brittany"}

\subsection{Scope del progetto}

\subsection{Descrizione ad alto livello del sistema}
Brittany è un sistema suddiviso in micro servizi:
\begin{itemize}
	\item \textbf{Edge}: Si tratta del componente che manipola i sensori e gli attuatori all’interno degli ambienti della serra. In un ambiente possono esserci n Edge;
	\item \textbf{Greenhouse Core}: Si tratta di un ambiente in cui degli agenti hanno un loro lifecycle, agiscono sugli Edge e contattano i Service qualora necessario.
	\item \textbf{Auth Service}: Si tratta del servizio di autenticazione, che permette di effettuare le operazioni di registrazione e login al sistema.
	\item \textbf{Settings Service}: Il servizio che consente di creare, modificare o eliminare determinate impostazioni utilizzate all’interno di Grennhouse Core, come ad esempio ogni quanto tempo campionare determinati tipi di dato.
	\item \textbf{Persistence Service}: Il servizio che permette di salvare e recuperare dati campionati.
	\item \textbf{Web Client}: Il servizio che offre possibilità ad un client di interagire con il sistema.
\end{itemize}
Greenhouse core deve necessariamente essere collocato nella stessa rete LAN in cui si trovano gli apparecchi di tipo Edge.

\centeredImage{img/architecture.png}{Architettura del sistema "Brittany"}{0.7}

\noindent La parte del dominio che è interessante da rappresentare e formalizzare è quella inerente ai dispositivi di tipo "Edge", ossia la parte più Core dell'intero sistema. \newline
Anche i parametri di configurazione di cui si occupa "Settings Service" sarebbero stati interessanti, ma sono già stati modellati accuratamente utilizzando altre tecnologie e sono stati quindi per ora esclusi.

\centeredImage{img/represent.png}{Parte del sistema "Brittany" presa in considerazione per il progetto corrente}{0.7}


