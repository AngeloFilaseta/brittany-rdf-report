\section{Introduzione}
Brittany è un software per semplificare ed automatizzare la gestione di un sistema di serre. L'obiettivo è quello di permettere all'utente di gestire da remoto le condizioni ambientali di una o più serre e di ricevere degli avvisi quando qualcosa non funziona in modo corretto.\newline

\noindent Il sistema, anche se completo, può essere ampliato e migliorato. Scopo di questo progetto è rappresentare parte della struttura del dominio e le varie relazioni tra i componenti, che infatti non sono attualmente mantenute in nessun modo. Questo passo permetterebbe un'integrazione tra i due progetti per rendere il sistema più stabile, preciso e semplice.\newline

\noindent Le tecnologie che sono state utilizzate sono RDF (Resource Description Framework) ed RDF Schema, in particolare si è scelta una rappresentazione in formato Turtle per una maggiore concisione.\newline

\noindent L'Ubiquitous Language del sistema è stato già ben definito per alcuni concetti. Scopo iniziale è stato quello di realizzare una buona struttura di base senza riutilizzare ontologie già esistenti. Una volta ottenuta una struttura sufficientemente solida sono stati collegati concetti presenti su altre ontologie disponibili online.\newline

\noindent Infine sono stati utilizzati strumenti aggiuntivi per garantire una buona qualità generale del progetto.
