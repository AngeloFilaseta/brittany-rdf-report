\section{Riuso di ontologie esistenti}
\subsection{DBpedia}
DBpedia contiene Linked Open Data in formato RDF. Le informazioni sono estratte dalle pagine di Wikipedia~\cite{wwwdbped0:online}.

\noindent In questo progetto DBpedia è stata utilizzata per reperire concetti piuttosto semplici, in particolare è stata utilizzata la classe Plant attraverso l'ontologia:
\begin{minted}{turtle}
@prefix dbo: <http://dbpedia.org/ontology/> .

dbo:Plant .
\end{minted}
È stata utilizzata anche una risorsa il cui tipo è proprio \textit{dbo:Plant}:
\begin{minted}{turtle}
@prefix dbr: <http://dbpedia.org/resource/> .

dbr:Basil .
\end{minted}

\subsection{domOS Common Ontology}
Una delle ontologie più utilizzate in questo progetto, in quanto ricca di contenuti compatibili con il dominio di "Brittany".
L'obiettivo di domOs Common Ontology è semplificare l'interoperabilità di dati e servizi all'interno di edifici nel dominio dell'IoT~\cite{domOSCom57:online}. L'ontologia quindi contiene risorse utili riguardo componenti hardware ed il modo in cui vengono utilizzati in ambienti reali:

\begin{minted}{turtle}
@prefix dco: <https://w3id.org/dco#> .

<#Environment>
rdfs:subClassOf dco:Space .

<#Edge>
owl:equivalentClass dco:Device .
\end{minted}

\subsection{FOAF}
FOAF è un progetto nato con lo scopo di collegare persone ed informazioni usando il web~\cite{FOAFVoca16:online}.

\noindent In questo progetto è stata utilizzata principalmente la proprietà \textit{foaf:name} per fornire nomi a risorse:
\begin{minted}{turtle}
@prefix foaf: <http://xmlns.com/foaf/0.1/> .

<#angelo-greenhouse>
a <#Greenhouse> ;
foaf:name "Angelo's Greenhouse" .
\end{minted}

\subsection{Hypermedia Controls Ontology}
Un'ontologia utilizzata per link, form e contenuti multimediali nel Web~\cite{Hypermed84:online}.

\noindent Questa ontologia viene utilizzata per specificare alcune proprietà delle InteractionAffordance contenute nel Thing Descriptor. In particolare viene utilizzata per specificare il tipo di contenuto che viene scambiato durante la connessione e a quale path è presente l'interazione:
\begin{minted}{turtle}
@prefix hctl: <https://www.w3.org/2019/wot/hypermedia#> .

<#angelo-action-turn-off-digital-light-sx>
	td:hasForm [
		hctl:forContentType "application/json" ;
		hctl:hasTarget "angelo-td-mock-ip/turnOff?id=angelo-digital-light-sx"
	] ;
\end{minted}

\subsection{JSON Schema}
Un vocabolario che contiene standard per la struttura di oggetti JSON~\cite{JSONSche6:online}.

Questa ontologia è utile per specificare i tipi all'interno delle InteractionAffordance del Thing Descriptor.\newline
\noindent Anche alcune caratteristiche degli Operation Handler come la struttura dei dati di input e di output sono rappresentati utilizzando alcuni di questi elementi:
\begin{minted}{turtle}
@prefix jsonschema: <https://www.w3.org/2019/wot/json-schema#> .

<#angelo-action-turn-off-digital-light-sx>
td:hasOutputSchema [ a jsonschema:StringSchema ] .
\end{minted}
\noindent In questo caso si specifica che il tipo di output di questa azione è di tipo stringa.\newline
\noindent Per gli Operation Handler sarebbe stato possibile utilizzare direttamente un tipo di dato diverso e più semplice, ma grazie a JSON Schema è possibile fornire da subito una descrizione più accurata di quali range di valori sono possibili.\newline
\noindent Inoltre il processo di inferenza risulta molto più semplice rimanendo consistenti ed utilizzando sempre le stesse classi.\newline

\subsection{Machine-to-Machine Measurement (M3)}
L'ontologia Machine-to-Machine Measurement (M3) è un linguaggio unificato, un dizionario ricco di nomenclature utili per identificare concetti nel dominio dell'IoT (Sensors, RFID tags, etc.)~\cite{SWoTSema18:online}.

\noindent In questo progetto è stata utilizzata proprio per meglio collegare concetti come sensori, attuatori ed altri tipi di hardware:

\begin{minted}{turtle}
@prefix m3: <http://sensormeasurement.appspot.com/m3#> .

<#SensorHw>
owl:equivalentClass m3:Sensor .
\end{minted}

\subsection{SEAS Ontology}
L'ontologia SEAS comprende anche una tassonomia su edifici, stanze e spazi in generale~\cite{SEAS:online}.

\noindent Il principale utilizzo è stato proprio per modellare meglio i concetti di serra e ambiente:

\begin{minted}{turtle}
@prefix seas: <https://w3id.org/seas/> .

<#Greenhouse>
owl:equivalentClass seas:Greenhouse .
\end{minted}

\subsection{W3 Thing Descriptor}
Un Thing Descriptor (TD, o W3C WoT Thing Descriptor) è un modello di rappresentazione royalty-free basato sul linguaggio JSON ed utilizzato nell Internet of Things (IoT). Il TD fornisce un modo univoco per descrive un oggetto o un servizio, in funzione delle sue capacità, dei protocolli che usa, ed altri metadati. Usare un Thing Descriptor semplifica notevolmente l’integrazione di dispositivi IoT all’interno di sistemi che ne fanno uso~\cite{ThingDes54:online}.

\noindent Anche se lo standard utilizzato per la creazione di un Thing Descriptor è il linguaggio JSON, è possibile una rappresentazione utilizzando qualsiasi linguaggio con sintassi RDF valida:
\begin{minted}{turtle}
@prefix td: <https://www.w3.org/2019/wot/td#> .

<#angelo-td-mock-ip>
a <#BrittanyThingDescriptor> ;
foaf:name "angelo-td-mock-ip" ;
td:title "Angelo's ESP8266 TD" ;
td:hasActionAffordance <#angelo-action-turn-off-digital-light-sx>,
                       <#angelo-action-turn-off-digital-light-dx>,
                       <#angelo-action-turn-on-digital-light-sx>,
                       <#angelo-action-turn-on-digital-light-dx> ;
td:hasPropertyAffordance <#angelo-property-is-on-digital-light-sx>,
                         <#angelo-property-is-on-digital-light-dx> .
\end{minted}
\subsection{Web of Things (WoT) Security Ontology}
Questa ontologia contiene concetti utilizzati per definire il tipo di sicurezza utilizzato all'interno del Thing Descriptor~\cite{WebofThi54:online}.

\noindent Per una configurazione priva meccanismi di sicurezza si può per esempio dichiarare un individuo come segue:
\begin{minted}{turtle}
@prefix wotsec: <https://www.w3.org/2019/wot/security#> .

<#nosec-configuration>
a wotsec:NoSecurityScheme ;
td:hasConfigurationInstance <#no-sec> .
\end{minted}
