\section{Quality Assurance}
In questa sezione vengono descritti alcuni strumenti utilizzati per il controllo della qualità del progetto.

\subsection{Pre-commit hooks}
I Git hook sono script che vengono eseguiti per identificare potenziali problemi (funzionali o stilistici) all'interno di file sorgenti prima che vengano versionati.\newline

\noindent Per questo progetto sono stati utilizzati classici hook stilistici:
\begin{itemize}
	\item controllo della presenza di un carattere newline a fine file;
	\item controllo sulla presenza di spazi inutili;
	\item molti altri, ad esempio per controllare che non siano presenti informazioni sensibili come chiavi private o file troppo grossi.
\end{itemize}

\noindent L'hook più importante però è \textit{gradle-task} che permette di lanciare il comando \textit{gradle test} prima di ogni commit.
Per ora il task di test non fa altro che lanciare un ulteriore comando per validare il file RDF.

\subsection{Gradle Task: test}
Il task di gradle \textit{test} permette di validare il file RDF:
\begin{minted}{bash}
$ gradle test
\end{minted}
La validazione del file RDF avviene tramite un comando lanciato dal task di gradle:
\begin{minted}{shell}
$ turtle --validate brittany.ttl
\end{minted}
Il comando è equivalente a:
\begin{minted}{shell}
$ riot --validate --syntax=ttl brittany.ttl
\end{minted}
Sia \textit{riot} che \textit{turtle} sono dipendenze del task, presenti nel framework Apache Jena~\cite{ApacheJe75:online}.

\subsubsection{GitHub Workflow}
Il workflow di GitHub è strutturato nel seguente modo:
\begin{enumerate}
	\item Checkout Repository;
	\item Validazione RDF.
\end{enumerate}

\noindent La validazione RDF avviene attraverso l'azione fornita da \textit{vemonet/jena-riot-action}~\cite{vemonetj43:online}. L'azione lancia esattamente lo stesso comando illustrato nella sottosezione precedente.
