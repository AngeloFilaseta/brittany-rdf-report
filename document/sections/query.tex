\section{SPARQL Query}
\subsection{Metodo utilizzato}
Le query sviluppate funzionano correttamente solo sull'ontologia che contiene anche tutte le triple che sono state inferite oltre a quelle inserite.

\subsubsection{Gradle Task: query}
\noindent È possibile eseguire le query attraverso Protege, oppure attraverso il task di gradle creato appositamente:

\begin{minted}{bash}
$ gradle query -Parg=QUERY_NAME

# es: $ gradle query -Parg=plant_list
\end{minted}

\noindent Per poter eseguire il task sono necessarie le dipendenze Apache Jena~\cite{ApacheJe75:online} e Pellet~\cite{AURenpel94:online}.
\subsection{Query sviluppate}
È possibile consultare il sorgente delle query. I file sono contenuti all'interno della cartella \textit{"query"}. Seguono una breve descrizione e degli esempi di output per ciascuna.
\subsubsection{plant\_list}
\paragraph{Descrizione}
Elenca tutte le Piante contenute in ogni ambiente Per ogni Serra.
\centeredImage{img/plant\_list.png}{Esempio di output per la query: plant\_list}{0.65}

\subsubsection{plant\_count\_env}
\paragraph{Descrizione}
Stampa il numero di piante presenti in ogni ambiente, insieme alla serra in cui esse sono situate.
\centeredImage{img/plant\_count\_env.png}{Esempio di output per la query: plant\_count\_env}{0.65}

\subsubsection{plant\_count\_gh}
Stampa il numero di piante presenti in ogni serra.
\centeredImage{img/plant\_count\_gh.png}{Esempio di output per la query: plant\_count\_gh}{0.38}

\subsubsection{edge\_location}
Stampa la lista di tutti gli Edge del sistema e per ciascuno è affiancato il Monitorable su cui sono piazzati.
\centeredImage{img/edge\_location.png}{Esempio di output per la query: edge\_location}{0.55}

\subsubsection{componentHw\_location}
Stampa una lista in cui compaiono tutti i Componenti Hardware. Per ciascuno viene riportato dove sono piazzati, seguendo la catena: \textit{edge} $\rightarrow$ optional(\textit{plant}) $\rightarrow$ \textit{environment} $\rightarrow$ \textit{greenhouse}.
\centeredImage{img/componentHw\_location.png}{Esempio di output per la query: componentHw\_location}{1}
