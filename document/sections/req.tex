\section{Requisiti}

\subsection{Requisiti Funzionali}

I requisiti funzionali riguardano le funzionalità che il sistema deve mettere a disposizione all’utente.

\noindent Il principale requisito funzionale consiste nel rappresentare la struttura del dominio del software "Brittany" e le varie relazioni tra i componenti, che infatti non sono mantenute in nessun modo. Una definizione rigorosa permetterebbe una più semplice gestione e catalogazione dei componenti, in particolare la rappresentazione vorrebbe permettere di:
\begin{enumerate}
	\item creare, modificare, cancellare una serra;
	\item creare, aggiungere e rimuovere ambienti ad una serra;
	\item aggiungere e rimuovere piante da ambienti;
	\item personalizzare dispositivi denominati "edge" in modo che possano monitorare ed effettuare azioni su una pianta o un ambiente;
	\item creare e definire i comportamenti dei moduli che potranno essere utilizzati per personalizzare i dispositivi edge.
	\item definire quali operazioni può eseguire ciascun modulo e su quali componenti hardware all'interno della serra.
\end{enumerate}

\subsection{Requisiti Non Funzionali}
I requisiti non funzionali riguardano le funzionalità che il sistema non deve necessariamente possedere per fare in modo che sia funzionante e corretto.\newline

\noindent La rappresentazione del sistema deve essere di facile intuizione, documentata e descritta accuratamente, in modo che sia semplice da utilizzare.\newline

\noindent Devono essere rispettate le convenzioni delle tecnologie che verranno usate.

\subsection{Requisiti Implementativi}

I requisiti di implementazione vincolano l’intera fase di realizzazione del sistema, ad esempio richiedendo l’uso di uno specifico linguaggio di programmazione e/o di uno specifico tool software.

\noindent Per lo sviluppo di questo progetto si è vincolati ad utilizzare le tecnologie:
\begin{itemize}
	\item \textbf{RDF}, \textbf{RDF Schema} e \textbf{OWL}: per la modellazione del dominio;
	\item \textbf{SPARQL}: per l'interrogazione dell'ontologia creata;
	\item \textbf{SWRL}: per esprimere regole di logica aggiuntive.
\end{itemize}
